\chapter{Einleitung}\label{ch:intro}

Wichtige Frage: Inwiefern lässt sich die Philosophie mit den Entwicklungsentscheidungen untermauern? Ist es lohnenswert, GO als Sprache weiter zu verfolgen?

\section{Motivation}
Go ist eine aufkommende Sprache, allerdings noch von den etablierten entfernt. GO wird weitreichend vor allem im Cloud Bereich verwendet. Hier kann Go trotz wenig neuen Ansätzen durch Fokus auf das ganze Entwicklungsumfeld brillieren.\cite{cox_go_2022}

%https://pypl.github.io/PYPL.html
\section{Warum GO}
In der Website pypl werden die Google Trends für die Programmiersprachen dargestellt. Es wird dargestellt, wie häufig nach einem Tutorial für die jeweiligen Programmiersprache im Vergleich mit anderen Programmiersprachen gesucht wird. In dieser kommt Go mit 2,07 Prozent auf Platz 12 Stand Februar 2025. Dies ist mit der höchste Stand auf dem GO jemals war. Dieser Wert ähnelt den Vorjahren mit 2,2 Prozent in Februar 2024, 1,9 Prozent Februar 2023 und 1,2 Prozent Februar 2022. 
%https://madnight.github.io/githut/#/pull_requests/2024/1
%https://innovationgraph.github.com/global-metrics/programming-languages
Weitere Statistiken, die GO in einem positiven Licht darstellen, sind von dem CodeBase Anbieter GitHub zu finden. Dies ist der größte Anbieter dieser Art und macht Daten zwecks Programmiersprachen Trends teilweise öffentlich. Ein Beispiel dafür ist der Github Innovation Graph. Dieser untersucht die Aktivität von Entwicklern für den Anbieter Github. Hier landet Go Stand 2024 auf Platz 16, welches die schlechteste Platzierung ist seit der Graph in 2020 gestartet wurde. Hier verliert Go ebenfalls gegen Java, Python und C++. Rust kommt in dem Graph auf Platz 21. 
Mit GitHut 2.0 werden ebenfalls die Aktivitäten von Entwicklern, die Github benutzten, dargestellt. Hier kann GO sich über die letzten Jahre öfters über C++ setzen, verliert aber weiterhin gegen Python und Java. 

Insgesamt lässt sich sagen,dass ein solides und steigendes Interesse an GO existiert. Dieses kommt hingegen nicht gegen die älteren wie Python Java und C++ ran. Go kann mit den ebenfalls neueren Programmiersprachen wie Rust mithalten. 

Ebenso zeigen diese Statistiken die Vielzahl an Programmiersprachen auf, die genutzt werden können. Die richtige Wahl bei der Festlegung auf eine Sprache birgt eine Herausforderung, welche große Auswirkungen haben kann. Eine Herangehensweise an das Problem wäre die initiale Konzeption zu beleuchten. Bei neueren Sprachen wie GO kann so vermieden werden, sich den gleichen und bekannten Frustrationen der populären Sprachen auszusetzen. Gleichen sich die Vorstellungen der Entwickler für eine Programmiersprache mit den eigenen und stimmt die Umsetzung mit der Vorstellung überein, erleichtert dies die Entscheidung. Somit ist eine genauere Untersuchung notwendig, um diese beiden Aspekte herauszuarbeiten. 

\section{Anwendungsgebiet von GO}
GO wird oft für Cloud Services benutzt. Go's Priorisierung von Skalierbarkeit, Einfachheit und Performanz kommt Cloud Computing zu gute. Ebenso der frühe Fokus auf paralleles Arbeiten ist hier relevant. Cloud Computing erfreut sich immer höherer Relevanz aufgrund von technischen Entwicklungen. Viele Unternehmen benötigen Hardware. Besonders Tech Unternehmen erfordern teils hohe Rechenleistungen. Da diese aber eine hohe Investition voraussetzen, kann mit Cloud Anbietern dynamischer agiert werden. Die benötigten Rechenleistungen können dynamisch festgelegt werden und bieteen kleineren Unternehmen einen einfacheren Einstieg. Da mit vielen ressourcenintensiven Themen gehandhabt wird wie KI, ist Cloud Computing relevant. 
%https://www.researchgate.net/profile/Srinivas-Jagirdar/publication/255994786_CLOUD_COMPUTING_BASICS/links/0c96052159b1a04dac000000/CLOUD-COMPUTING-BASICS.pdf
\section{Zielsetzung und Vorgehensweise}
Das Ziel ist ein besseres Verständnis der Programmiersprache Go zu schaffen. Es soll der grundlegende Aufbau der Sprache untersucht werden. Es soll die Philosophie und die Entwicklungsentscheidungen herausgearbeitet und anhand von Code untermauert werden. Es soll überprüft werden, ob diese beiden Elemente übereinstimmen und GO somit hält was es verspricht. Es sollen fehlende Funktionen hervorgehoben werden, die es vielleicht davon abhält, noch weiter etabliert zu sein. Es soll ein Ausblick gegeben werden, ob es lohnenswert ist, weiter diese Sprache zu verfolgen.

Zu Beginn muss herausgearbeitet werden, welche Aspekte für eine Programmiersprache relevant sind. Zentral für diese Entscheidung ist die Monographie von Donovan \cite{donovan_go_2016}. Diese beleuchtet alle aus Autorensicht relevanten Aspekte und bilden die Grundlage für die Entscheidung. Weiterhin werden diese Aspekte genauer betrachtet. Wichtig ist hier sowohl die generelle Funktionsweise als auch die Art und Weise wie diese implementiert ist. Dazu wird der Code genauer betrachtet.