\chapter{Einleitung}\label{ch:intro}

\section{Motivation}
Go ist eine aufkommende Sprache, allerdings noch von den etablierten entfernt. GO wird weitrechend vorallem im Cloud Bereich verwendet. Hier kann Go trotz wenig neuen Ansätzen durch Fokus auf das ganze Entwicklungsumfeld brillieren.\cite{cox_go_2022}
\\\\ https://pypl.github.io/PYPL.html In der Website pypl werden die Google Trends für die Programmiersprachen dargestellt. In dieser kommt Go mit 2,07 Prozent auf Platz 12 Stand Februar 2025. Dieser Wert ähnelt den Vorjahren mit 2,2 Prozent in Februar 2024, 1,9 Prozent Februar 2023 und 1,2 Prozent Februar 2022. Diese Google Trends zeigen ein solides Interesse an GO, kommt hingegen nicht gegen die älteren ran.
\\\\ Ebenso zeigen Google Trends die Vielzahl an Programmiersprachen auf, die gentzt werden können. Die richtige Wahl bei der Festlegung auf eine Sprache birgt eine Herausforderung, welche große Auswirkungen haben kann. Eine Herangehensweise an das Problem wäre die initiale Konzeption zu beleuchten. Bei neueren Sprachen wie GO kann so vermieden werden, sich den gleichen und bekannten Frustrationen der populären Sprachen auszusetzen. Gleichen sich die Vorstellungen der Entwickler für eine Programmiersprache mit den eigenen und stimmt die Umsetzung mit der Vorstellung überein, erleichtert dies die Entscheidung. Somit ist eine genauere Untersuchung notwendig, um diese beiden Aspekte herauszuarbeiten. 

\section{Zielsetzung und Vorgehensweise}
Es soll die Sprache auf Funktion untersucht werden. Es soll die Philosophie und die Entwicklungsentscheidungen herausgearbeitet werden und anhand von Code untermauern. Es soll überprüft werden, ob diese beiden Elemente übereinstimmen und GO somit hält was es verspricht. Es sollen fehlende Funktionen hervorgehoben werden, die es vielleicht davon abhält, noch weiter etabliert zu sein. Da viele Programmiersprachen existieren soll ein Ausblick gegeben werden, ob es lohnenswert ist weiter diese Sprache zu verfolgen.
\\\\
Zu Beginn muss herausgearbeitet werden, welche Aspekte für eine Programmiersprache relevant sind. Zentral für diese Entscheidung ist die Monographie von Donovan \cite{donovan_go_2016}. Diese beleuchtet alle aus Autorensicht relevanten Aspekte und bilden die Grundlage für die Entscheidung. Weiterhin werden diese Aspekte genauer betrachtet. Wichtig ist hier sowohl die generelle Funktionsweise als auch die Art und Weise wie diese implementiert ist. Dazu wird der Code genauer betrachtet.
