\chapter{Grundlagenkapitel}\label{ch:data}

\section{Geschichte von Go}
Go wurde 2007 als Nebenprojekt in Google gestartet. Nach einer zweijährigen Entwicklungsphase wurde sie 2009 als Open Source Projekt veröffentlicht. Die Idee und das Konzept wurde aus einer Unzufriedenheit in der Firma Google herausgearbeitet. Besonders Kompilierzeiten und die Skalierbarkeit wiesen aus Sicht einiger Google Softwareentwickler Probleme aus. Zu der Zeit wurden mehrere der populären Sprachen wie C++, Java und Python verwendet. Da diese älteren Sprachen vor der aufkommenden Multicore Entwicklung konzipiert wurden, bestand für Google ein Handlungsbedarf. Diese die Entwicklung richtig ausnutzen zu können stellten Entwickler vor Herausforderungen. Aufgrund von undurchsichtigen Dependencies wurde ein unabsichtliches Neuladen von Libraries ein größeres Problem. Die Kompilierzeiten von C++ wuchsen sowohl in Dauer als auch in Speicherbedarf. Ebenso hochskaliert musste die Zusammenarbeit zwischen Entwicklern. Als einer der größten Techfirmen zu der Zeit stellte Google mehrere tausend Mitarbeiter für die Softwareentwicklung an. Diese mussten an einer Code Basis effizient arbeiten können. Code zu Standardisieren wurde zu einer hohen Priorität und nahm so Einfluss auf die frühen Entwicklungsschritte. Go ist aufgrund dieses Hintergrunds nicht nur bei Google die populäre Wahl für Cloud Infrastruktur. Auch andere Firmen wie Docker setzen auf die Sprache.\cite{cox_go_2022}
\cite{meyerson_go_2014}

\section{Ziel/Philosophie von Go}
Zentral für Go ist die Einfachheit. Die Sprache ist aus Frustrationen mit etablierten Sprachen entstanden, die aufgrund von langen Entwicklungshistorien sehr viel Komplexität eingeführt haben. Deshalb soll GO unter allen Umständen einfach gehalten werden, auch wenn es dadurch zu langsameren und weniger effizienten Programmen führen kann.\cite{donovan_go_2016}
\\\\Weiterhin ist jedoch eine schnelle und effiziente Ausführung wichtig. Ebenso die Skalierbarkeit hardwaretechnisch als auch Entwicklungstechnisch soll gewährleistet sein. Viele Entwickler sollen gleichzeitig auf mehreren Maschinen mit viel Kapazität arbeiten können. Es wird versucht aus der Entwicklung älterer Sprachen die richtigen Schlüsse zu ziehen und deren Probleme zu vermeiden. Statt mit neuen und umfangreichen Ansätzen will Go mit Balance und Fokus auf das Entwicklungsumfeld überzeugen.\cite{cox_go_2022}

Charakteristische Merkmale: 

Wie ist diese Philosophie umgesetzt(Schauen in Donovan Buch inwiefern dort Sachen erläutert sind):
Ein Garbage Collector erhöht den Speicherbedarf und senkt die Geschwindigkeit. Dieser gestaltet Programme einfacher und wird deshalb in Go benutzt. Ebenso verbessert sich die Skalierbarkeit indem der Compiler Anpassungen automatisch vornimmt und das Programm nicht angepasst werden muss. Generell übernimmt der Compiler soll Auf Vererbung wird verzichtet da dies eine komplexere Klassen Implementation benötigt. Ersetzt werden kann diese durch Interfaces, die dynamischer gestaltet werden können als in anderen Sprachen. 
Concurrency ist auch irgendwie anders.
Type System ist einfacher gehalten und hat dadurch aber auch weniger Möglichkeiten. Ist aber recht ausgeglichen mit genügend Komplexität aber trotzdem einfach \cite{donovan_go_2016}.