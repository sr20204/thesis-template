\makeatletter\@openrightfalse
\chapter{Textzusammenfassung}\label{ch:txt}
Im ersten Kapitel in der Monografie von Cutajar\cite{cutajar_learn_2024} werden die Einsatzmöglichkeiten von Parallelität thematisiert. Diese bietet eine Möglichkeit, mehrere CPU-Kerne auf einmal zu verwenden. Standardmäßig werden Aufgaben eines Programms nach\-einander ab\-gearbeitet. Mit Parallelität kann zwischen diesen Aufgaben gewechselt werden. Der Autor zieht Vergleiche zu Beispielen aus dem realen Leben, in der Parallelität meist unter\-bewusst benutzt wird, um die Wichtigkeit zu verdeutlichen. Es werden sowohl Vorteile als auch Szenarien genannt, in denen Parallelität notwendig ist.\cite{cutajar_learn_2024}
\\
\\ Mit der Entwicklung von Computer-Hardware wird eine dieser Notwendigkeiten dargestellt. Aufgrund von physikalischen Einschränkungen kann die Hardware nicht mehr stark vertikal skaliert werden. Das Potenzial von einzelnen Maschinen und CPU-Kernen sind aufgrund von Kühlungsproblemen bereits weitgehend ausgereizt. Mehrere Kerne, die koordiniert werden müssen, sind zum Standard geworden und ermöglichen eine horizontale\- Skalierung. Die Einführung von Cloud-Computing ist ein Ergebnis dieser Skalierung. Mehrere Computer werden dezentral dazugeschaltet. Alle diese Hardware muss mit Parallelität koordiniert werden. \cite{cutajar_learn_2024}
\\
\\Wenn nur ein Kern existiert, kann trotzdem von Parallelität profitiert werden. Oft muss auf Serverantworten oder Nutzereingaben gewartet werden. Diese Wartezeiten können sinnvoll genutzt werden. Ebenso muss nicht jede Aufgabe komplett durchgeführt werden. Bei Nutzereingaben kann unterbrochen und erst eine Antwort zurückgegeben werden. Eine Anwendung kann somit immer reaktiv bleiben. Als Beispiele werden Online-Shops und Schreibprogramme genannt. \cite{cutajar_learn_2024}
\\
\\In Go wurde Parallelität von Anfang an berücksichtigt und in die Sprache eingebaut. Das Ziel ist diese effizient, gut lesbar und gut benutzbar zu machen. Der Fokus liegt für den Programmierer auf dem Schreiben von korrekten Programmen durch Benutzen von Goroutinen. Der Compiler von Go kümmert sich selbst um die effiziente, parallele Ausführung. Go kümmert sich auch um die Verwaltung von Ressourcen. Dies erlaubt eine einfache Skalierung, da für den Programmierer keine zusätzliche Last anfällt. Koor\-dinierung der Goroutinen wird mit der Datenstruktur eines Channels erreicht. Dies stellt eine der Möglichkeiten dar, Parallelität zu implementieren. Die andere Möglichkeit ist das Synchronisieren mithilfe von Mutexen. Welche Methode sich besser eignet, unterscheidet sich von Fall zu Fall. Go stellt beides zur Verfügung.\cite{cutajar_learn_2024}
\@openrighttrue\makeatother 
