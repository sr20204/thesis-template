\makeatletter\@openrightfalse
\chapter{Priorisierung der Literatur}\label{ch:litprio}
Die Monographie Donovan\cite{donovan_go_2016} ist wichtig, da sie sehr allgemein gehalten ist und dennoch\- wich\-tige Einblicke in die Programmier\-sprache Go bietet. Als Thema sollen einige Aspekte von Go genauer beleuchtet werden. Allerdings ist die genaue Definition einer Pro\-gram\-mier\-sprache nicht eindeutig. Mehrere Aspekte können beleuchtet werden. Die Priorisierung von Themengebieten ist eine unabdingbare Voraussetzung für den Einstieg in das Thema. Diese Monographie bietet eine gute Grundlage dafür. Es existieren Aspekte\mbox, die in jedem Fall behandelt werden müssen. Diese bilden Grundlagen, die in jeder Programmier\-sprache vorhanden sein sollten.  Ein wichtiger Grundbaustein einer Programmiersprache sind Datentypen. Diese werden in Kapitel 3 ausführlich behandelt. Datenstrukturen nehmen ebenfalls einen wichtigen Platz ein. In Kapitel 4 werden alle relevanten Datenstrukturen behandelt.\cite{donovan_go_2016}
\\ 
\\Die Monographie Cutajar\cite{cutajar_learn_2024} hingegen behandelt ein spezifisches Thema. Programmieren mit Parallelität benutzt den Ansatz von Multithreading, das in anderen Programmiersprachen existiert. Es existieren jedoch kleinere Unterschiede weshalb genaueres Betrachten notwendig ist. Besonders wichtig für die Grundlagen von Parallelität sind Kapitel 2,3 und 4. Ebenso relevant ist Kapitel 9, welches das Programmieren mit Channels behandelt. Diese stellen eine Datenstruktur dar, die in der Art nur in Go existiert. Sie dienen als Kommunikator zwischen Threads und stellen somit ein wichtiges Werkzeug für die Koordination dar.\cite{cutajar_learn_2024}
\@openrighttrue\makeatother 

